% ****** Start of file apssamp.tex ******
%
%   This file is part of the APS files in the REVTeX 4.1 distribution.
%   Version 4.1r of REVTeX, August 2010
%
%   Copyright (c) 2009, 2010 The American Physical Society.
%
%   See the REVTeX 4 README file for restrictions and more information.
%
% TeX'ing this file requires that you have AMS-LaTeX 2.0 installed
% as well as the rest of the prerequisites for REVTeX 4.1
%
% See the REVTeX 4 README file
% It also requires running BibTeX. The commands are as follows:
%
%  1)  latex aps25samp.tex
%  2)  bibtex apssamp
%  3)  latex apssamp.tex
%  4)  latex apssamp.tex
%
\documentclass[%
 reprint,
nofootinbib,
%superscriptaddress,
%groupedaddress,
%unsortedaddress,
%runinaddress,
%frontmatterverbose,
%preprint,
%showpacs,preprintnumbers,
%nofootinbib,
%nobibnotes,
%bibnotes,
aps,
%pra,
%prb,
%rmp,
%prstab,
%prstper,
%floatfix,
]{revtex4-1}



\usepackage[utf8]{inputenc}
\usepackage[english]{babel}
\usepackage{dsfont}
\usepackage{amsmath}
\usepackage{ mathrsfs }
\usepackage{amssymb}
\usepackage{graphicx}% Include figure files
\usepackage{dcolumn}% Align table columns on decimal point
\usepackage{bm}% bold math
\usepackage{amsmath}
\usepackage{varioref}
\usepackage{booktabs}
\usepackage[bottom]{footmisc}
\usepackage{minted} % for pseudocode

\usepackage{physics}
\usepackage[ruled,vlined]{algorithm2e}
\usepackage{algpseudocode}
\usepackage{listings}

\usepackage{booktabs}

\usepackage{tikz}
\usepackage{float}
\usepackage{siunitx}



\newcolumntype{C}{>{$}c<{$}}
\AtBeginDocument{
\heavyrulewidth=.08em
\lightrulewidth=.05em
\cmidrulewidth=.03em
\belowrulesep=.65ex
\belowbottomsep=0pt
\aboverulesep=.4ex
\abovetopsep=0pt
\cmidrulesep=\doublerulesep
\cmidrulekern=.5em
\defaultaddspace=.5em
}

\usepackage{ntheorem}
\newtheorem{theorem}{Teorem}
\newtheorem*{theorem-non}{Teorem}
\usepackage{hyperref}% add hypertext capabilities
%\usepackage[mathlines]{lineno}% Enable numbering of text and display math
%\linenumbers\relax % Commence numbering lines

%\usepackage[showframe,%Uncomment any one of the following lines to test
%%scale=0.7, marginratio={1:1, 2:3}, ignoreall,% default settings
%%text={7in,10in},centering,
%%margin=1.5in,
%%total={6.5in,8.75in}, top=1.2in, left=0.9in, includefoot,
%%height=10in,a5paper,hmargin={3cm,0.8in},
%]{geometry}

% \renewcommand{\vec}[1]{\mathbf{#1}} %ny definisjon av \vec så det blir bold face i stedet for vector-pil.

\begin{document}


\title{En videnskablig analyse af den maritime tommelfingerregel for vurderingen af kollisionskurs ved betragtning af omgivelsernes relative bevægelse til et andet fartøj}
\author{Mikkel Metzsch Jensen}

\affiliation{Department of Physics, University of Oslo\\}
\date{\today}


\begin{abstract}
  In this article we investegate the rule of thumb for recognizing when two boats are on a collision course. The proposed statement to be analyzed, \textit{The background method}, can be formulated shortly as: "If the static background seen behind the oncoming boat does not move relatively to the boat, then you are on a collision course.". This turned out to be strongly connected to the dertermination of the relative bearing. By mathematical proof we found the relative bearing to remain constant when on collision course, and therefore this is a precise way of recognizing colloision course. On the other hand we say that the background method can be used as mean to determine if the relative bearing is constant in special cases. This were mainly derived by an analytical analysis but is supported by numerical simmulations. In general the background method is found to be valid when used in greater distances to the coast, losely estimated to approximately 800 meters or more. This is found to be dependent on the value of relative bearning and the angle between the heading and the coast. This is explained in more detailes on figure \ref{fig:limit_dimension} and \ref{fig:limit_coastdis}.
\end{abstract}

\maketitle


\onecolumngrid

\begin{figure}[H]
  \centering
  \includegraphics[width=0.8\linewidth]{figures/eksempel1.pdf}
  \caption{forside billede}
  \label{fig:forside}
\end{figure}

\twocolumngrid



\hfill
\clearpage
% \newpage


\section{Introduktion}
Som søfarer findes der en række nødvendige og basale regler som sikrer god orden og sikkerhed til søs. Heriblandt har vi vigereglerne som regulerer færdslen på vandet og forebygger, at skibe støder sammen \cite{respektforvand}. Disse regler beskriver som udgangspunkt hvordan man skal agere hvis man er på kollisionskurs med andre fartøjer. Hertil er det fordelagtigt for søfaren at blive opmærksom på en mulig kollision i så god tid som mulig. Denne rapport udsprigner af en diskussion med Lars Juel Hansen (sommeren 2019) om en forslået tommelfingerregel for netop denne vurdering. Formuleringen for det vi fremadrettet skal kalde \text{baggrundsmetoden} lyder som følger:
\begin{quote}
Betrakt båden med mulig kollisionskurs og noter et fastliggnede punkt i baggrunden som synes bag båden. Hvis dette baggrundspunkt i den efterfølgende tid lader til at flytte sig i forhold til båden, da er du ikke på kollisionskurs med båden. Forbliver baggrundspunktet i sigtelinjen bag båden er dette et tegn på kollisionskurs.
\end{quote}
Ved en undersøgelse af tilgængelige internetkilder finder vi lidt forskellige gengivelser af denne regel. Ifølge \cite{duelighed} kan man ved "betydelige afstande" til kysten bruge observationen om hvordan kollisionsfartøjet trækker "over land" til at vurdere pejlingen (Retning fra iagttageren til den genstand, der pejles \cite{ordbog}). Altså hævdes det at hvis den modsejlende trækker mod højre (styrbord) i forhold til baggrunden så vil fartøjet gå højre om iagterens fartøj. Modsat vil det gå venstre (bagbord) om iagterens fartøj hvis det trækker til venstre over land. Hvis ikke fartøjet bevæger sig kendeligt i forhold til baggrunden, vil det i overenstemmelse med baggrundsmetoden efter sigende indikere at fartøjerne er på kollisionskurs. Dog er det værd at bemærke at denne metode indføres som et middel til at vurdere pejlingen. I andre kilder som \cite{studienoter}, \cite{retsinformation} og \cite{groensund}, er det netop pejlingen som angives som en direkte indikator på kollisionskurs. I dette tilfælde kan man altså bruge en sammenligning af det andet fartøjs bevægelse relativt til et fast punkt på eget skib som en alternativ metode til baggrundsmetoden.
\\
I denne rapport skal vi fremføre en matematisk analysere af baggrundsmetodens pålidelighed som indikator for kollisionskurs. Dette gøres ved først at beskrive pejlingens betydning for denne vurdering. Med dette udgangspunkt skal vi undersøge baggrundsmetodens præcision i forhold til at vurdere pejltræk og dermed udpege eventuelle begrænsninger.

\subsubsection{Målgruppe og brug af matematisk notation}
Rapporten bestræber sig på at give en fyldestgørende beskrivelse af det omtalte problem samtidig som dette formidles til en målgruppe uden særpræget matematisk baggrund. Dette er ikke mulig på alle punkter, da en ordenlig bevisførelse kræver en hvis mængde matematik. Jeg har dog forsøgt at gengive de underliggende resultater undervejs, således at man kan springe beregninger og udledninger over uden at miste kontekst. Det gælder særlig udledningen i afsnit \ref{sec:pejling_betydning}.




\section{Metode}
\subsection{Definering af problemet}
Vi forestiller os to både til søs, som nærmer sig hinanden. Vi kalder den ene båd for hovedbåden (HB), hvor vi har placeret iagtageren, som ønsker at vurdere om bådene er på kollisionskurs. Den anden båd kalder vi for kollisionsbåden (KB). Vi antager at begge bådene bevæger sig med konstant hastighed, dvs. retlinjet og med konstant fart. Vi beskriver hver båd som et enkelt punkt i et to-dimensionsjonalt aksesystem som tilsvarer bådens position på vandets overflade. Vi bruger aksetitlerne x og y til at beskrive positionen $\vec{P} = (x, y)$ . Positionen til hver båd bliver da en funktion af tid, som bestemt af parametrene: Startposition $\vec{P}_0$ og hastighed $\vec{v}$. For de to både har vi bevægelseslingingen
\begin{align}
  \vec{P}(t) =  \vec{P}_0 + \vec{v}\cdot t
  \label{eq:motion}
\end{align}
Vi definerer en kollision som tilfællet hvor bådene har samme position til samme tidspunkt. Bemærk at en kollision i praksis vil ske i flere tilfælle da vi ikke har taget hensyn til bådenes udstrækning her. Dette regnes dog for en ubetydelig detalje for dette problem.\\
For at vudere om bådene er på kollisionskurs er det angiveligt nyttigt at benytte pejlingen. Vi definerer i matematisk forstand pejlingen $\theta_{HB}$ fra HB som vinklenen mellem kursen til HB og sigtelinjen fra HB til KB. Dette er illustreret på figur \ref{metode_tegning}.

\begin{figure}
  \includegraphics[width=\linewidth]{figures/metode_tegning.png}
  \caption{}
  \label{fig:metode_tegning}
\end{figure}

\subsection{Numeriske simuleringer som støtte til matematiske resultater}
I tillæg til den matematiske analyse vil vi bruge numeriske metoder som som støtte til resultaterne. Da alle bevægelserne er lineære kan vi simmulere bådenes præcise bevægelse uden usikkerhed. Dette gøres ved at evaluere bevægelsesligning \ref{eq:motion} ved forskellige tidspunkter. Dette er analogt til eulers metode uden acceleration. Vi skriver simmuleringskoden i python som ser ud som vist nedenfor.

\begin{minted}[breaklines, breakautoindent = true]{python}
import numpy as np

def simulator(MB_start, OB_start, MB_end, OB_end, T = 10, dt = 0.1):
    K = int(T/dt + 1)         #Steps
    MB_pos = np.zeros((K,2))  #MB = Main Boat
    OB_pos = np.zeros((K,2))  #OB = Other Boat
    t = np.zeros(K)           #Time

    #Initial position
    MB_pos[0] = MB_start
    OB_pos[0] = OB_start

    #Calculate constant velocity
    MB_vel = (MB_end - MB_pos[0])/T
    OB_vel = (OB_end - OB_pos[0])/T

    #Main update loop
    for k in range(K - 1):
        MB_pos[k+1] = MB_pos[k] + MB_vel*dt
        OB_pos[k+1] = OB_pos[k] + OB_vel*dt
        t[k+1] = t[k] + dt
    return MB_pos, OB_pos
\end{minted}


\section{Resultater}

\subsection{Pejlingens betydning for kollisionskurs (matematisk udledning)}\label{sec:pejling_betydning}
Da vi har antaget at bådene bevæger sig med konstant hastighed kan vi bruge en lineær transformation til at skifte koordinatsystem til intertialsystemet hvor HB er i ro (HB's intertialsystem). Med andre så kan vi frit vælge at beskrive positionen til KB som den opleves for en iagtager ombord på HB. Positionen $P'_{KB}$ til KB i det nye intertialsystem kan skrives:
\begin{align*}
  \vec{P'}_{KB}(t) &= \vec{P}_{KB}(t) - \vec{P}_{HB}(t) \\
  &= \vec{P}_{0,KB} + \vec{v}_{KB}t - \vec{P}_{0,HB} - \vec{v}_{HB}t \\
  &= (\vec{P}_{0,KB} - \vec{P}_{0,HB}) + (\vec{v}_{KB} - \vec{v}_{HB})t \\
  &= \vec{P'}_{0,KB} + \vec{v'}_{KB}t
\end{align*}
Vi ser at $P'_{KB}$ også beskriver en retlinjet bevægelse. I tilfællet hvor bådene er på kollisionskurs ved vi at $P'_{KB}(t_k) = 0$ ved kollisionstidspunktet $t_{k}$. Dette medfører sammenhængen:
\begin{align}
  \vec{P'}_{0,KB} &= - \vec{v'}_{KB}t_k \nonumber \\
  \begin{pmatrix} x'_{0,KB} \\ y'_{0,KB} \end{pmatrix}\frac{1}{t_k} &=   -\begin{pmatrix} v'_{x,KB} \\ v'_{y,KB} \end{pmatrix}
  \label{eq:P=v}
\end{align}
Vi kan da finde pejlingen $\theta_{HB}$ ved at omrksive $P'_{KB}$ til polære koordinater. Her er vi bare interesseret i vinkelkoordinat $\phi_{KB}$ som kan bestemmes som
\begin{align*}
  \phi_{KB}(t) &= \arctan{\left( \frac{y'(t)}{x'(t)}\right)} \\
  &= \arctan{\left( \frac{y'_{0,KB} + v'_{y,KB}t}{x'_{0,KB} + v'_{x,KB}t}\right)}
\end{align*}
Vi bruger da sammenghængen fra ligning \ref{eq:P=v} og finder
\begin{align*}
  \phi_{KB}(t) &= \arctan{\left( \frac{y'_{0,KB} - y'_{0,KB}\frac{t}{t_k}}{x'_{0,KB} - x'_{0,KB}\frac{t}{t_k}}\right)} \\
  &= \arctan{\left(\frac{y'_{0,KB}}{x'_{0,KB}} \frac{1 - \frac{t}{t_k}}{1 - \frac{t}{t_k}}\right)} \\
  &= \arctan{\left(\frac{y'_{0,KB}}{x'_{0,KB}}\right)} = \text{konst.} \\
\end{align*}
Fra dette ser vi at vinkelkoordinat $\phi_{KB}$ er konstant (uavhængig af tid), hvilket medfører at pejlingen også er konstant:
\begin{align*}
  \theta_{HB} &= \frac{\pi}{2} - \phi_{KB} = \text{konst.}
\end{align*}
Fra dette ræsonoment har vi altså vist at pejlingen vil være konstant i tilfællet hvor bådene er på kollisionkurs. Hvis bådene følger kollisionskursen men i modsat retning (bevæger sig væk fra hianden), kan vi indføre betingelsen $P'_{KB}(t_i) = 0$ for et tidspunkt $t_i < 0$. Derved kan vi opstille en ligning analog til \ref{eq:P=v} og derved finde at pejlingen også vil være konstant i dette tilfælle. Hvis ikke bådene er på kollisionskurs er ligning \ref{eq:P=v} ikke længere gyldig og $P'_{KB}$ kan tage hvilken som helst retlinjet bane. Det betyder at $\phi_{KB}$ og dermed også at pejlingen $\theta_{KB}$ ændre sig som funktion af tid. Dette fører til slutningen:
\begin{theorem}
  Hvis og bare hvis to både har kollisionskurs og nærmer sig i afstand vil pejlingen fra den ene båd til den anden være konstant.
  \label{Teo:pejling}
\end{theorem}
\subsubsection{Gengivelse af konstant-pejling-udledningen uden matematisk notation}
Siden begge bådene antages at bevæge sig med konstant fart og i en ret linje vil en iagtager på HB også se at KB bevæger sig med konstant fart og i en ret linje. I tilfællet hvor bådene er på kollisionskurs vil en iagtager på HB altså se at KB har kurs direkte mod iagteren. Dette er den eneste mulige kurs hvorpå bådene kan kollidere uden at medparterne ændrer retning eller fart. Derfor følger det at pejlingen også vil være konstant i tilfællet med kollisionskurs. Det eneste andet tidspunkt at man vil opleve at pejlingen er konstant er hvis begge bådene står stille eller bevæger sig i modsat retning af hvad der kræves for kollisionskurs.



 \subsection{Grænsebetingelser for brug af baggrundsmetoden}
 Med udgangspunkt i teorem \ref{Teo:pejling} kan vi undersøge om brugen af baggrundensmetoden er en pålidelig indikator for en fremtidig kollision. Vi skal altså undersøge om det er sammenfald mellem en konstant pejling og tilfællet hvor baggrunden ikke bevæger sig relativt til sigtelinjen gennem KB. \\
 I tilfællet med kollisionskurs ved vi fra teorem \ref{Teo:pejling} at sigtelinjen fra HB gennem KB vil have en konstant vinkel i forhold til kystlinjen. I det enkle tilfælle hvor kystlinjen er relinjet og parallel med kursen til HB vil sigtelinjens skæring med kystlinjen forflytte sig med samme hastighed som HB. Dette resultat bekræftes ved simuleringen vist på figur \ref{fig:eks1}.
 \begin{figure}[H]
   \includegraphics[width=\linewidth]{figures/eksempel1.pdf}
   \caption{En simulering af bådene HB og KB på kollisionskurs, med kollisionspunkt 100 meter fra kysten. Her plottes fire positioner for bådene (inklusiv kollisionspunktet) jævnt fordelt i tid. Fra de sorte stiblede linjer ser vi hvordan sigtelinjen fra HB gennem KB skærer med kysten. Dette skæringspunkt forflytter sig som forventet i henhold til HB's bevægelse.}
   \label{fig:eks1}
 \end{figure}
 Hvis denne forflytning er kendelig, på trods af at bådene er på kollisionskurs og pejlingen er konstant, vil baggrundsmetoden være vildledene. Fra HB's perspektiv vil baggrundspunktet (BP) bevæge sig med en hastighed $\vec{v'}_{BP} = - \vec{v}_{HB}$. Dette giver en vinkelhastighed $\omega_O$:
 \begin{align*}
   \omega_O = -\frac{v_{HB}\sin{(\theta_{HB})}}{d}
 \end{align*}
 hvor $v_{HB} = |\vec{v}_{HB}|$ er bådens fart, $\theta_{HB}$ er pejlingen og $d$ er afstanden til kysten via sigtelinjen. Hvis vi definerer minimumsgrænsen for en kendelig forflytning som vinkelhastigheden $\omega_k$ får vi at kriteriet for anvendelse af baggrundsmetoden er
 \begin{align}
   |\omega_O| &< \omega_k \nonumber \\
   \frac{v_{HB}\sin{(\theta_{HB})}}{d} &< \omega_k \nonumber \\
   \frac{v_{HB}\sin{(\theta_{HB})}}{\omega_k} &< d
   \label{eq:limit}
 \end{align}
Vi kan da bruge denne sammenhæng  (\ref{eq:limit}) til at finde Den minimume afstand $d$ for brug af baggrundsmetoden som funktion af pejlingnen $\theta_{HB}$. Dette resultat er vist på figur \ref{fig:limit_dimensionless}.
 \begin{figure}[H]
   \includegraphics[width=\linewidth]{figures/limit_dimensionless.pdf}
   \caption{Den minimume afstand $d$ som funktion af pejlingen $\theta_{HB}$ angivet med dimensionløse enheder. For at baggrundsmetoden skal være anvendelig må $d$ ligge over kurven vist her.}
   \label{fig:limit_dimensionless}
 \end{figure}


\subsubsection{Definering af kendelig forflyting}
For at bestemme enhederne til figur \ref{fig:limit_dimensionless}, må vi definere hvad en kendelig forflytning er. Hertil er vi nødsaget til at lave en række kvalificerede gæt for at finde et estimat for $\omega_k$. \\
Vi kan forestille os at vi ved observation af KB vil benytte et centralt punkt på båden som referanse mod baggrunden. En mulig defintion på en kendelig forflytning kan da være at baggrundspunktet har flyttet sig ud til eller forbi kanten af båden i løbet af en observationsperiode. Observationsperioden estimeres til 10 sekunder, svarende til den nedre grænse af tidsintervallet 10-20 sekunder som nævnes i \cite{duelighed}. Vi siger da at forflytningen er kendelig hvis baggrundspunktet har flyttet sig en relativ afstand større eller lig halvdelen af bådens tværlængde set fra iagtageren i løbet af 10 sekunder. For sejlbåde i den lidt større klasse kan vi bruge en gennomsnitlig bredde på 5 m og længde på 15 m. Siden vi ønsker at finde tværlængden set fra iagtageren kan vi bruge 10 m som et middelestimat (tilsvarer 45 graders sigtelinje). Til sidst må vi vurdere den gennomsnitlige aftsand hvorved en søfarer har brug for at vurdere kollisionsrisikoen. Hertil bruges 200 meter som en minimumsværdi. Ud fra dette finder vi at en kendelig forflytning kan defineres ved en vinkelhastighed større eller lig $\omega_k$:
\begin{align}
  \omega_k = \frac{\arctan{(\frac{1}{2}\frac{10 \text{m}}{200 \text{m}})}}{10 \text{s}} = 0.0025 s^{-1} \approx  0.14^{\circ}s^{-1}
  \label{eq:omega_k}
\end{align}
Bemærk at vi ved valg af den nedre grænse for observationsperioden, et stort estimat for bådens tværlængde og et minimumsestimat for afstanden mellem bådene ved observation,  effektivt estimerer den øvre grænse for en kendelig forflytning. Dette vil føre til at estimatet for den gyldige afstand $d$ bliver et minimumsestimat og dermed giver de bedst tænkelige vilkår for baggrundsmetodens gyldighed. \\
Vi kan bruge at en typisk bådhastighed er 4 knob hvilket tilsvarer omtrent 2 m/s. Med disse værdier kan vi skalere resultatet fra figur \ref{fig:limit_dimensionless} sådan at vi får et fysisk enhed på resultatet. Det nye resultat er vist på figur \ref{fig:limit_dimension}
\begin{figure}[H]
  \includegraphics[width=\linewidth]{figures/limit_dimension.pdf}
  \caption{dimensionsløs.}
  \label{fig:limit_dimension}
\end{figure}
I en situation hvor den retlinjede kyst har en vinkel $\beta$ i forhold til HB's kurs kan vi omregne afstanden $d$ langs sigtelinjen til den korteste afstand $s_{kyst}$ mellem HB og kysten. Omregning gøres som
\begin{align*}
  s = d\cdot \sin{(\theta_{HB} + \beta)}
\end{align*}
Ved at anvende denne omregning får vi resultatet vist på figur \ref{fig:limit_coastdis}.
\begin{figure}[H]
  \includegraphics[width=\linewidth]{figures/limit_coastdis.pdf}
  \caption{}
  \label{fig:limit_coastdis}
\end{figure}


\section{Diskussion}
Ud fra den matematiske slutning om pejlingen i teorem \ref{Teo:pejling} er det klart at pejlingen kan bruges som et direkte indikator på kollisionskurs. Hvis pejlingen ikke ændres og det andet fartøj nærmer sig er dette ensbetydene med en kollision, hvis ikke kurs eller retning ændres. Som antydet i \cite{duelighed} (se introduktion) kan baggrundsmetoden bruges som en indikator på pejltræk. Dette er dog ikke uden begrænsninger da baggrunden, i tilfælde med konstant pejling, vil flytte sig tilsvarende bådens egen relative bevægelse i forhold til baggrunden. Siden denne bevægelse bliver mindre og mindre tydelig desto længere væk fra den observerede baggrund man kommer, kan baggrundes metoden effektivt bruges ved større afstand til kysten. Dette er i overenstemmelse med påstanden fra \cite{duelighed}. Med kendskab til den minimumme vinkelhastighed $\omega_k$ for en kendelig forflytning samt HB's hastighed, vil man kunne kortlægge det gyldige område for baggrundsmetoden via resultaterne fra figur \ref{fig:limit_dimensionless}. Med kvalificerede gæt på disse størrelser kom vi frem til resultaterne i figur \ref{fig:limit_dimension}. Fra dette finder vi at baggrundsmetoden kræver den største afstand $d \approx 800$ m når pejling er på 90 $^{\circ}$. For andre pejlinger var dette estimat mindre. Det er klart at dette er et usikkert estimat og det bør derfor kun bruges som en indikator på størrelsesordnen for hvilket område hvor baggrundsmetoden er gyldig. På figur \ref{fig:limit_coastdis} får vi relevant information om hvad afstand $d$ tilsvarer i den direkte afstand til kysten $s_{kyst}$ som vil være mere brugbar i praksis. Denne afhænger af vinklen mellem kursen og kysten, mne hvis vi sammenfatter det mulige interval som funktion af pejlingen får vi datapunkterne vist på tabel \ref{tab:valid_area}.
\begin{table}[H]
  \begin{center}
  \caption{....}
  \begin{tabular}{|c|c|} \hline
  $\theta_{HB}$ [$^{\circ}$] & $s_{kyst}$-interval [m]  \\ \hline
  30 & [208, 402] \\ \hline
  60 & [361, 725]  \\ \hline
  90 & [0, 833] \\ \hline
  \end{tabular}
  \label{tab:valid_area}
  \end{center}
\end{table}
Dette giver et bedre billede af den minimumme direkte aftsand til kysten som kræves for at baggrundsmetoden kan anvendes. \\
\\



% Noter:\\
% \\
% Bemærk at vi i praksis må tage hensyn til at båden har en hvis udstrækning og at den dermed også kolliderer når punkterne passerer tæt forbi hinanden. Dette har dog ikke betydning for den teortiske model, og ved andvendelse af modellen i praksis må man bruge resultaterne i overensstemmelse med en ønsket sikkerhedsradius ved forbipassering. (Se diskussion for mere info om dette.)
% \\
% Baggrundsmetoden kan sandsynligvis anvendes tættere på kysten end den estimerede grænse hvis søfaren evner at tage højde for baggrundes forflytning grundet relativ forflytning til kysten
% \\
% Bestemmelsen af $\omega_k$ er et meget løst estimat. Angiver størrelsesordenen for anvendelig område.
% \\
% Klart at en mere direkte vurdering af pejlingen er fordelagtig.
% \\
% "Pejltræk til forenden af et stort skib eller til et skib med slæb er ikke tilstrækkeligt. Der skal være pejltræk til den agterste kant i sejlretningen." \cite{groensund}.
% \\
% Se bort for strømforhold, vind og andre mærkelige ting?

\section{Konklusion}
Fra den matematiske udledning kan vi konkludere at pejlingen kan bruges som et direkte indikator på om to både er på kollisionskurs. Dette kan formuleres som teoremet:

\begin{theorem-non}
  Hvis og bare hvis to både har kollisionskurs og nærmer sig i afstand vil pejlingen fra den ene båd til den anden være konstant.
\end{theorem-non}

Videre fandt vi at baggrundsmetoden, hvor man vurdere om baggrunden flytter sig relativt til den anden båd, kan bruges som en indirekte indikator på kollisionskurs. Dette skyldes at baggrundens forflytning kan være kendelig selvom bådene er på kollisionskurs, når iagtageren befinder sig relativt nær kysten. Et estimat af grænseværdien for denne kystafstand er vist på figur \ref{fig:limit_coastdis} og tabel \ref{tab:valid_area}, men vi ser at generelt at baggrundsmetoden kan bruges for kystafstande på ca. 800 m eller større.
\\
Baggrundspunktet bevæger sig ligesom dit eget skib inde på kysten. Så spørgsmålet er om du ville kunne se en kendelig forflytning af dit eget skib (med sinus led af $\theta_{HB}$) hvis det var placeret på baggrundspunktet.





\begin{thebibliography}{9}
  \bibitem{respektforvand} Repspekt for vand \url{https://www.respektforvand.dk/paa-havet/laer-at-sejle/vigeregler} (sidst læst: 16/01/2021)
  \bibitem{duelighed} Duelighed.dk. Date. Edition. Skipper-kursus (slide 03-02), tilgængelig ved \url{http://www.duelighed.dk/tutorial_soevejsregler/03_02.htm} (sidst læst: 05/01/2021)
  \bibitem{ordbog} Hjemmeværnet: Maritime udtryk, tilgængelig ved \url{https://www.hjv.dk/oe/HVF122/Sider/Maritime-udtryk.aspx} (sidst læst: 05/01/2021)
  \bibitem{studienoter} Søren Toftegaard O. (2013), \textit{LystSejlads}, s. 19 (afsnit 1.4) , tilgængelig ved \url{http://studienoter.dk/Sejlads/Noter/LystSejlads.pdf} (sidst læst: 05/01/2021)

  \pagebreak
  \bibitem{retsinformation} retsinformation.dk: \textit{Bekendtgørelse om søvejsregler} 20/11/2009. Regel 7: fare for sammenstød (d) (20/11/2009), tilgængelig ved \url{https://www.retsinformation.dk/eli/lta/2009/1083} (sidst læst: 05/01/2021)
  \bibitem{groensund} Albrechten S. (2007). \textit{Sejlads for Begyndere} \url{http://www.groensund.dk/upl/website/sejlads/SejladsforBegyndere2.pdf}
  \bibitem{wiki:eye} Wikipedia: Naked eye \url{https://en.wikipedia.org/wiki/Naked_eye} (sidst læst: 11/01)
\end{thebibliography}

\clearpage
\onecolumngrid
\section*{Appendix}

\subsection{Kollisions simmuleringer}

\begin{figure}[H]
  \includegraphics[width=\linewidth]{figures/subplot_C1.pdf}
  \caption{}
  \label{}
\end{figure}


\begin{figure}[H]
  \includegraphics[width=\linewidth]{figures/subplot_C2.pdf}
  \caption{}
  \label{}
\end{figure}

\clearpage
\subsection{Ikke-kollisions simmuleringer}
\begin{figure}[H]
  \includegraphics[width=\linewidth]{figures/subplot_NC1.pdf}
  \caption{}
  \label{}
\end{figure}





\end{document}
